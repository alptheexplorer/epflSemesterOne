\documentclass{article}
\usepackage[utf8]{inputenc}
\usepackage{amsmath}
\usepackage{tcolorbox}
\usepackage{amssymb}
\usepackage[
top    = 2.50cm,
bottom = 2.50cm,
left   = 2.75cm,
right  = 2.75cm]{geometry}
\usepackage{fancyhdr}
\pagestyle{fancy}
\lhead{programming notes}
\rhead{EPFL/Alp Ozen}

\newtheorem{thm}{Theorem}
\date{\vspace{-5ex}}
\title{ProgrammingBasics}

\begin{document}
\maketitle

\section{Variables and types}
Some of the basic variable types in java are \textbf{int} and \textbf{double}
To declare a variable, it is initialized through an identifier along with a type such as:

\begin{verbatim}
    int number; 
\end{verbatim}

One must be careful when using equality as:
\begin{verbatim}
    a = b; /* copies b into a*/
    b = a; /* copies a into b*/
\end{verbatim}

Also assignment like this are possible:

\begin{verbatim}
    a = a + 1;
\end{verbatim}

Here, we are simply saying that store the new value of a as whathever a is an increment 1. 
\\
Another important type is the \textbf{final} type used to declare a constant. 
\\
If we want to swap values we do the following:
\begin{verbatim}
    int a = 1;
    int b = 2;
    int temp;
    
    temp = 2;
    a = b;
    b = temp;
\end{verbatim}
\\
A thread is a single line of execution within a program. 

\section{Access and non-access modifiers}

\subsection{Access modifiers}

\begin{enumerate}
    \item public, (world,class,subclass,package)
    \item protected, (class,subclass,package,subclass outside package(inheritance)
    \item default, (class,subclass)
    \item private, (class)
\end{enumerate}


\end{document}